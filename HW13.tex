\documentclass[11pt,a4paper]{article}
\usepackage[margin=2.5cm]{geometry}                % See geometry.pdf to learn the layout options. There are lots.
%\geometry{letterpaper}                   % ... or a4paper or a5paper or ... 
%\geometry{landscape}                % Activate for for rotated page geometry
%\usepackage[parfill]{parskip}    % Activate to begin paragraphs with an empty line rather than an indent
\usepackage{graphicx}
\usepackage{amssymb}
\usepackage{epstopdf}
\usepackage{amsmath}
\usepackage{caption}
\usepackage{array}
\usepackage{hyperref}
\usepackage{wasysym}
\usepackage{amsthm}
\usepackage{halloweenmath}


\newcommand{\Z}{\mathbb{Z}}


\begin{document}
\begin{Large}
\centerline{\bf MAT 218 - Elementary Number Theory (Spring 2)}\medskip
\centerline{\bf Homework 13}\medskip
\end{Large}
{\bf Due:} Tuesday, May 18, at 11:59 PM CDT on Gradescope. 

{\bf Special directions:}  All problems must be done in \LaTeX. 


\hrulefill

\begin{enumerate}
	
	
	
	\item Solve the following congruences. You will receive 1 point for each part. Explain how you solved the problem. You will not get credit if you merely plugged in value(s) until you found (a) solution(s).
		\begin{enumerate}
			\item \(6x \equiv 9 \mod 21\)
			\item \(4x \equiv 9 \mod 101\)
			\item \(3x \equiv 17 \mod 51\)
			\item \(15x \equiv 25 \mod 50\)
		\end{enumerate}
	
	
	\item (\S 3.1 \#5). We can count the total number of possible congruences of the form \(ax \equiv b \mod 24\) to be \(23 \cdot 24\) since we can pick \(23\) values for \(a\) and \(24\) values for \(b\). 
		\begin{enumerate}
			\item How many of these possible congruences have at least one solution?
			\item How many of these possible congruences have a unique solution?
		\end{enumerate} 
	
	\item (\S 3.1 \#7). Use the congruence \(612x \equiv 156 \mod 84\) to  find all integer solutions to the equation \(612x+ 84y= 156\).
	
	\item \textbf{(This question must be done on your own.)} (Review exercise) Find all positive integers \(n\) so that the ring of residues \(Z_{n}\) has exactly \(6\) units.
	
	\item (Review exercise) Let \(p\) be an odd prime and \(k\) be a positive integer.  Which \(a \in Z_{p^{k}}\) have the property that that \(a^{2} \equiv 1 \mod p^{k}\)?
	
\end{enumerate}

\end{document}