\documentclass[11pt,a4paper]{article}
\usepackage[margin=2.5cm]{geometry}                % See geometry.pdf to learn the layout options. There are lots.
%\geometry{letterpaper}                   % ... or a4paper or a5paper or ...
%\geometry{landscape}                % Activate for for rotated page geometry
%\usepackage[parfill]{parskip}    % Activate to begin paragraphs with an empty line rather than an indent
\usepackage{graphicx}
\usepackage{amssymb}
\usepackage{epstopdf}
\usepackage{amsmath}
\usepackage{caption}
\usepackage{array}
\usepackage{hyperref}
\usepackage{wasysym}
\usepackage{amsthm}
\usepackage{halloweenmath}


\newcommand{\Z}{\mathbb{Z}}


\begin{document}
\begin{Large}
\centerline{\bf MAT 218 - Elementary Number Theory (Spring 2)}\medskip
\centerline{\bf Homework 12}\medskip
\end{Large}
{\bf Due:} Friday, May 14, at 11:59 PM CDT on Gradescope.

{\bf Special directions:}  All problems must be done in \LaTeX.

{\bf Name:} Yang (Garry) Gao - Sorry I am not submitting this assignment on time. I was making up HW11 earlier and sincerely apologize for the inconvenience I caused. 

\hrulefill

\begin{enumerate}



	%1
	\item Suppose you toss a \(6\)-sided die \(10\) times and record the number on the top of the die each time.  Use Inclusion-Exclusion to determine the number of ways those dice could be thrown so that each of the \(6\) numbers occur at least once in your list of \(10\) numbers.  Here we assume tossing a \(1\) and then nine \(6\)'s is different than tossing nine \(6\)'s first and then a \(1\).

	By using permutation algorithm, we know that there are $6^{10}$ permutations in total. And then we need to exclude the permutations that do not contain all the numbers. There are $6 \choose i$ ways to exclude $i$ numbers, and for each way, there are $(6-i)^{10}$ permutations. Thus using the Inclusion-Exclusion, the number of permutations that contains all six numbers would be \[\sum_{k=0}^{6} (6-k)^{10} {6 \choose k} (-1)^k = 16435440.\]
	The number of ways those dice could be thrown as the question indicates is 16435440.

	%2
	\item Use Inclusion-Exclusion to determine the number of permutations of the set \(\left\{1,2,\ldots,9\right\}\) in which at least one odd integer is fixed.

	Recall: A permutation can be thought of as an arrangement of a list of \(n\) elements. For example, for \(n= 3\) the \(6\) permutations are:

	\[1,2,3 \hspace{0.8cm} 1,3,2 \hspace{0.8cm} 2,1,3 \hspace{0.8cm} 2,3,1 \hspace{0.8cm} 3,1,2 \hspace{0.8cm} 3,2,1\]

	Here we say \(1,2,3\) and \(1,3,2\) both fix \(1\) (since \(1\) remains as the first element in the list), while \(1,2,3\) and \(3,2,1\) both fix \(2\) (since \(2\) remains as the second element of the list).

	By using permutation algorithm, we know that there are $9!$ permutations in total. And then we need to exclude the permutations that do not fix odd numbers. Since there are 5 odd integers in the set, there are $5 \choose i$ ways to fix $i$ odd numbers, and for each way, there are $(9-i)!$ permutations. Thus using the Inclusion-Exclusion, the number of permutations that fix at least one odd number would be \[\sum_{k=1}^{5} (9-i)! {5 \choose i} (-1)^{k+1} = 157824.\]
	The number of ways to fix at least one odd integer is 157824.

	%3
	\item (\S 2.6 \#1). Find the multiplicative inverse of \(5 \mod 16\) using Euler's theorem.

	Using Euler's theorem, since 5 is a unit, $5^{\phi(16)} \equiv 1$ mod $16.$ Hence $5^{\phi(16)} \equiv 5^8 \equiv 5(5^7) \equiv 1$ mod $16,$ and the residue class that is the inverse of 5 is represented by $5^7.$ $5^7 \equiv 13$ mod $16.$ The multiplicative inverse of 5 mod 16 is 13.

	%4
	\item (\S 2.6 \#8). Let \(p\) be a prime.  If \(a^{p} \equiv b^{p} \mod p \), prove that \(a \equiv b \mod p\).

	\begin{proof}
		By Corollary 2.6.2a, since $p$ is a prime, we have \(a^{p} \equiv a \mod p, b^{p} \equiv b \mod p.\) We also have \(a^{p} \equiv b^{p} \mod p \) from the assumption. In other words, there exisits integers $i, j, k$ so that \[pi=a^p-a; pj=b^p-b; pk=a^p-b^p.\] Then by doing some simple arithmetic, we can get \[p(k-i+j) = pk-pi+pj = a^p-b^p-a^p+a+b^p-b = a-b.\] By this we know that $p \mid a-b,$ and which means $a \equiv b \mod p.$
	\end{proof}

	%5
	\item (\S 2.6 \#11). Let \(p\) be a prime. If \(a \equiv b \mod p\), prove that \(a^{p} \equiv b^{p} \mod p^{2}\).

	\begin{proof}
		By Corollary 2.6.2a, since $p$ is a prime, we have \(a^{p} \equiv a \mod p, b^{p} \equiv b \mod p.\) We also have \(a \equiv b \mod p \) from the assumption. In other words, there exisits integers $i, j, k$ so that \[pi=a^p-a; pj=b^p-b; pk=a-b.\] And then \[a^p-b^p = pi-pj+pk = p(i-j+k).\] In the meanwhile, using binomial theorem, we know that \[a^p-b^p = \sum_{k=0}^{p}{p \choose k}a^{p-k}(-b)^{k} - \sum_{k=1}^{p-1}{p \choose k}a^{p-k}(-b)^{k}.\]

		Sorry, I cannot solve this problem :(
	\end{proof}

	%6
	\item (\S 2.6 \#9) Find the remainder when \(6^{385}\) is divided by \(16\).

	We easily know that $16 \mid 6^5.$ And since $385=5\cdot 77, 6^{385}=6^5\cdot 6^{77}.$ By the transitive property of ``divides'', we know that $16 \mid 6^{385}.$ Thus $6^{385} \equiv 0$ mod $16.$




\end{enumerate}

\end{document}
