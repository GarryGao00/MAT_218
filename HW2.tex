\documentclass[11pt,a4paper]{article}
\usepackage[margin=2.5cm]{geometry}                % See geometry.pdf to learn the layout options. There are lots.
%\geometry{letterpaper}                   % ... or a4paper or a5paper or ... 
%\geometry{landscape}                % Activate for for rotated page geometry
%\usepackage[parfill]{parskip}    % Activate to begin paragraphs with an empty line rather than an indent
\usepackage{graphicx}
\usepackage{amssymb}
\usepackage{epstopdf}
\usepackage{amsmath}
\usepackage{caption}
\usepackage{array}
\usepackage{hyperref}
\usepackage{wasysym}
\usepackage{amsthm}


\newcommand{\Z}{\mathbb{Z}} %This is a user-defined command. This means that if you type ``\Z'' in a math environment, it will show up as the blackboard bold ``Z'' we use in class. In general, to get blackboard bold letters, you use the mathbb command. So, blackbord bold R (for real numbers) would be \mathbb{R} (done in a math environment!). But sometimes you want to use a symbol a lot and don't want to have to write out the full command/formatting every time. This is where user-defined commands become really helpful! User-defined commands are always placed in the preamble of the document (before ``\begin{document}'').

%Using a ``%'' sign starts a comment environment. Everything on the same line as the ``%'' and after it will be viewed as a comment, and will not affect the code or be visible. Certain characters like ``%'' and ``$'' are commands within LaTeX. If you want to use an actual symbol, use a ``\'' before it. For example, ``Salma did really well. She got 98\% on her exam. She deserves an extra \$20 for that!''

\begin{document}
\begin{Large}
\centerline{\bf MAT 218 - Elementary Number Theory (Spring 2)}\medskip
\centerline{\bf Homework 2}\medskip
\end{Large}
{\bf Due:} Friday, April 9, at 11:59 PM CST (UTC -6) on Gradescope.

{\bf Special directions:} n/a


\hrulefill

\begin{enumerate}
	\item (\S 1.2 \#7). Consider the table on page 8 of the book.  For what integers is \(\tau\!\left(n\right)\) odd?  What common property do those integers share? (Are you able to prove your answer? Later in the term, we will develop mathematical machinery to prove this more easily than you can now, but see if you can prove it with your current tools!)
	
	\item (\S 1.2 \#13). Write a definition for prime numbers which uses the function \(\sigma\!\left(n\right)\).  In particular, your definition should apply to all prime numbers and no composite numbers. (Are you able to prove your answer?)
	
	\item Explain, in your own words, why mathematical induction makes sense. (You do not have to prove anything for this problem.)
	
	\item (\S 1.3 \#4). Prove that \(3\) divides \(2^{2n}-1\) for every positive integer \(n\). 
	
	\item (\S 1.3 \#7). \textbf{(This question must be done on your own.)} If \(x \ne 1 \), prove that for every positive integer \(n\),
	
	\[ \frac{1-x^{n}}{1-x} = 1 + x + \ldots + x^{n-1}. \]
	
	\item Prove that for every positive integer \(n\), the value \(n^{3}+ 2n\) is divisible by \(3\).
	
	\item (\S 1.3 \#10). Prove that for any positive integer \(k\), there exists a sequence of $k$ consecutive composite integers.  As an example, for $k=3$, the sequence ``\(14, 15, 16\)'' is a sequence of \(k=3\) composite integers.
\end{enumerate}

\end{document}