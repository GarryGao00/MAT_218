\documentclass[11pt,a4paper]{article}
\usepackage[margin=2.5cm]{geometry}                % See geometry.pdf to learn the layout options. There are lots.
%\geometry{letterpaper}                   % ... or a4paper or a5paper or ... 
%\geometry{landscape}                % Activate for for rotated page geometry
%\usepackage[parfill]{parskip}    % Activate to begin paragraphs with an empty line rather than an indent
\usepackage{graphicx}
\usepackage{amssymb}
\usepackage{epstopdf}
\usepackage{amsmath}
\usepackage{caption}
\usepackage{array}
\usepackage{hyperref}
\usepackage{wasysym}
\usepackage{amsthm}
\usepackage{halloweenmath}


\newcommand{\Z}{\mathbb{Z}}


\begin{document}
\begin{Large}
\centerline{\bf MAT 218 - Elementary Number Theory (Spring 2)}\medskip
\centerline{\bf Homework 14}\medskip
\end{Large}
{\bf Due:} Thursday, May 20, at 11:59 PM CDT on Gradescope. 

{\bf Special directions:}  All problems must be done in \LaTeX. 


\hrulefill

\begin{enumerate}
	
	
	
	\item (Regarding \S 3.2). Let \(p\) and \(q\) be odd primes. In HW 10 problem 2, you showed that the square roots of \(1\) in the ring of residues \(Z_{pq}\) are \(1\), \(pq-1\), elements which are simultaneously one more than a multiple of \(p\) and one less than a multiple of \(q\), and elements which are simultaneously one more than a multiple of \(q\) and one less than a multiple of \(p\). Prove that there are exactly four square roots of \(1\) in \(Z_{pq}\). 

	\item (\S 3.2 \#5). Find the solution of the simultaneous congruences.
		\begin{align*}
			5x &\equiv 2 \mod 3 \\
			2x &\equiv 4 \mod 10 \\
			4x &\equiv 7 \mod 9 
		\end{align*}
	
	\item (\S 3.2 \#6). Fix nonzero integers \(a\) and \(b\), and let \(f(x) = ax+b\). For any positive integer \(k\), prove that there exists an integer \(n\) such that \(f(n)\) has at least \(k\) distinct prime divisors. (Hint: \(f(n)\) is divisible by \(p\) if and only if \(f(n) \equiv 0 \mod p\).) (Warning: there's something a bit subtle here.)
	
	\item (Review exercise) What are the last \textbf{two} digits of the decimal form of \(3^{404}\)? Prove your answer.
	
	\item \textbf{(This question must be done on your own.)} (Review exercise) How many positive integers less than or equal to \(100{,}000\) are multiples of \(4\), \(6\), or \(9\)? You may use a calculator to assist with your computation, but you must show your work.
	
	\item \textbf{(Bonus question!)} (This question is not required to be turned in. If you turn it in, you will get bonus points, depending on the rating you get.)  Let \(k\) be a positive integer.  Which \(a \in Z_{2^{k}}\) have the property that that \(a^{2} \equiv 1 \mod 2^{k}\)?

\end{enumerate}

\end{document}